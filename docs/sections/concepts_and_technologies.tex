\chapter{Concetti e tecnologie}

%%%%%%%%%%%%
% Content
%%%%%%%%%%%%
\section{Funzione hash}
Le funzioni hash crittografiche sono una serie di algoritmi che, dato in input una stringa di lunghezza idealmente arbitraria,
restituiscono in output una stringa di bit che rispetta le seguenti caratteristiche:

\begin{itemize}
    \item La lunghezza dell'output è costante
    \item Input identici producono output identici
    \item Un avversario con potenza di calcolo polinomiale che conosca l'output della funzione non è in grado di risalire l'input
    \item L'output deve essere ben distribuito, ovvero deve essere indistinguibile da una distribuzione casuale uniforme
    \item A piccole variazioni nell'input devono corrispondere grandi variazioni nell'output
\end{itemize}

Si noti che ottenere e verificare queste proprietà è un compito molto delicato,
per cui si fa affidamento ad algoritmi e librerie preesistenti con implementazioni sicure, efficienti e testate.


\section{Albero di Merkle}
\label{sec:merkle-tree}
Gli alberi di Merkle (\autoref{lab:merkle-tree}), inventati da Robert Bruce Merkle \cite{art:merkle}, sono un tipo di struttura dati con le seguenti caratteristiche:

\begin{itemize}
    \item Ogni foglia contiene l'hash dell'elemento associato
    \item Risalendo l'albero, ogni nodo contiene l'hash dei due figli
    \item La radice contiene l'hash dell'albero
\end{itemize}

\begin{figure}[h]
    \centering
    \begin{forest}
        for tree={draw,inner sep=10pt,l=20pt,l sep=20pt}
        [
        H(ABCD EFGH)
        [H(AB BC)
                [H(A B)
                        [H(A)]
                            [H(B)]
                    ]
                    [H(C D)
                        [H(c)]
                            [H(D)]
                    ]
            ]
            [H(EF GH)
                [H(E F)
                        [H(E)]
                            [H(F)]
                    ]
                    [H(G H)
                        [H(G)]
                            [H(H)]
                    ]
            ]
        ]
    \end{forest}
    \caption{Albero di Merkle \label{lab:merkle-tree}}
\end{figure}

Utilizzare un albero di Merkle per assicurare l'immutabilità dei dati è una soluzione molto efficace e sicura,
ed rappresenta una strategia molto comune nella realizzazione di architetture distribuite come la blockchain. \\
Un albero di Merkle, infatti, è in grado di provare l'immutabilità di un numero enorme di dati anche molto grandi
mantenendo in memoria solo la radice dell'albero. \\
Inoltre, è possibile verificare l'appartenenza all'albero di uno qualsiasi delle sue foglie rivelando unicamente
il valore della foglia incriminata e gli hash necessari a ricostruire la radice (\autoref{lab:merkle-tree-verify}).
Non è quindi necessario fornire informazioni sulle altre foglie.

\begin{figure}[h]
    \centering
    \begin{forest}
        for tree={draw,inner sep=10pt,l=20pt,l sep=20pt}
        [
        H(ABCD EFGH), fill=green!25
        [H(AB BC), fill=blue!25
        [H(A B)
        [H(A)]
            [H(B)]
        ]
        [H(C D)
        [H(c)]
            [H(D)]
        ]
        ]
        [H(EF GH), fill=green!25
        [H(E F), fill=green!25
        [H(E), fill=red!25]
        [H(F), fill=blue!25]
        ]
        [H(G H), fill=blue!25
        [H(G)]
        [H(H)]
        ]
        ]
        ]
    \end{forest}
    \caption{Per verificare l'appartenenza della foglia \colorbox{red!25}{E} all'albero,
        è necessario fornire unicamente gli \colorbox{blue!25}{hash} dei nodi fratelli.
        Con quelli è possibile ricostruire la \colorbox{green!25}{radice} dell'albero \label{lab:merkle-tree-verify}}
\end{figure}

\section{Metamask}
\label{sec:metamask}
MetaMask è un portafoglio software pensato per interagire con blockchain basate su Ethereum.
Permette a chiunque di partecipare molto comodamente alla blockchain attraverso un estensione browse o un'app mobile. \\
Nel concreto, MetaMask permette agli utenti di memorizzare le proprie chiavi private direttamente nell'estensione,
permettendo poi agli sviluppatori di interagire con le sue \gls{api}.
Ciò consente, ad esempio, di creare in maniera sicura transazioni da inviare agli altri nodi della blockchain. \\
Sebbene i più critici possano rimanere contrariati dalla fiducia riposta in questo tipo di software, la sua adozione
è sempre più comune.