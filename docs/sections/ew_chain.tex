\chapter{Energy Web chain}
\label{cap:ewc}

%%%%%%%%%%%%
% Content
%%%%%%%%%%%%
\section{Confronto con Ethereum}
Non è un mistero che la \gls{ewc} sia basata su Ethereum, più precisamente sulla sua ottava hard-fork, Istanbul \cite{wiki:ew-fork}.
È in programma una ulteriore migrazione alla hard-fork Berlino, che avverrà quanto prima \cite{wiki:ew-berlino}. \\
Anche alcuni dei servizi offerti, come l'\gls{ewns} sono fork con modifiche minime di progetti nati su Ethereum. \\

Tuttavia, ci sono alcuni aspetti in cui le due reti differiscono, come riassunto nella seguente tabella: \\

\hskip-.6cm
\begin{tabular}{||p{6cm}|p{7cm}||}
    \hline
    Problema                                                    & Modifica                                                                                                                    \\
    \hline\hline
    Basso throughput, alti costi, scalabilità limitata          & Utilizzo della \gls{poa}: incrementa il throughput fino a 30x                                                               \\
    \hline
    Poco adatta a piccoli dispositivi (IoT)                     & Maggiore focus sui \textbf{light client} per connettere anche piccoli dispositivi (\gls{iot})                               \\
    \hline
    Nessuna distinzione fra i nodi con diverse autorizzazioni   & È possibile differenziare fra nodi con compiti ed autorizzazioni diverse                                                    \\
    \hline
    Difficoltà a gestire transazioni che necessitino di privacy & Possibilità di mantenere i dati privati integrando protocolli crittografici e appoggiandosi a storage esterni, se richiesto \\[1ex]
    \hline
\end{tabular}

\section{Testnet}
Energy Web mette a disposizione una testnet che permette a chiunque sia interessato alle peculiarità della loro offerta di testare tutti i servizi senza spendere token reali. \\
La testnet si chiama Volta, e non è dissimile alle testnet che affiancano Ethereum. \\

Di seguito le differenze fra Volta e la \gls{ewc} \cite{wiki:ew-two-networks}:\\

\hskip-.6cm
\begin{tabular}{||p{4cm}|p{5cm} p{5cm}||}
    \hline
                       & Volta                                               & EWC                                                     \\ [0.5ex]
    \hline\hline
    Data di lancio     & Aprile 2019                                         & Giugno 2019                                             \\
    \hline
    Funzione primaria  & Pre-produzione                                      & Produzione                                              \\
    \hline
    Token              & 90M + 10M di compensi \newline transazioni e faucet & 90M + 10M di compensi \newline transazioni              \\
    \hline
    Tariffe            & Le risorse usate non hanno valore monetario         & Valore monetario delle transazioni in base al gas usato \\
    \hline
    Caratteristiche    & 5 secondi per block \newline limite di 8M gas       & 5 secondi per block \newline limite di 8M gas           \\
    \hline
    Connessione ad ETH & Bridge su Kovan Test Net                            & Bridge su Ethereum main-net con un ERC-20 token         \\
    \hline
    Nodi validatori    & 3 al lancio \newline max 150                        & 10 al lancio \newline max 150                           \\ [1ex]
    \hline
\end{tabular}

\section{Algoritmo di consenso Proof of Authority}
L'algoritmo che permette di assicurare un consenso sullo stato della blockchain fra tutti i nodi è \gls{poa}, più precisamente l'algoritmo Aura \cite{art:aura}\cite{wiki:poa}. \\
Nel caso di \gls{ewc}, i nodi validatori sono gestiti dai partner dell'associazione.
Si tratta, nella maggior parte dei casi, di aziende leader del settore energetico. \\
In breve, il funzionamento è il seguente:

\begin{enumerate}
    \item Tutti i nodi validatori possiedono una lista aggiornata degli altri nodi validatori, oltre alla copia completa della blockchain e ad alcuni meta-dati ad essa collegati (es. il throughput della blockchain)
    \item Per una finestra temporale ben definita, un validatore primario viene scelto dall'algoritmo e svolge il compito di raccogliere le transazioni e produrre il nuovo blocco. La scelta del validatore primario è in funzione del timestamp e del numero di validatori
    \item Se il validatore primario non riesce a produrre il blocco (es. problemi hardware) o il blocco non viene convalidato dagli altri nodi (es. problemi di connessione), il prossimo validatore primario riprenderà dalle transazioni rimaste in sospeso
    \item Gli altri validatori controllano che le transazioni siano legittime e firmano il blocco per poi propagarlo alla rete
    \item Se la maggioranza dei validatori ha verificato il blocco, questo è aggiunto alla blockchain
\end{enumerate}


\begin{figure}[ht]
    \includegraphics[height=14cm,keepaspectratio]{ew-poa}
    \centering
    \caption{Passi dell'algoritmo \gls{poa} \cite{img:ew-poa}}
    \label{lab:ew-poa}
\end{figure}

\subsection{Smart-contract di sistema}
Gli smart-contract di sistema sono una collezione di smart-contract già presenti su \gls{ewc} che implementano il protocollo di consenso \gls{poa}.
Il codice che li compone è pubblico e consultabile nella \href{https://github.com/energywebfoundation/ewc-system-contracts}{repository GitHub}.
Sintetizzando, vi sono tre smart-contract principali:

\begin{itemize}
    \item \href{https://energy-web-foundation.gitbook.io/energy-web/technology/the-stack/trust-layer-energy-web-chain/system-architecture/system-contracts/validator-set-contract#overview}{\textbf{Validator Set Contracts}}: gestisce i permessi e il comportamento del validatore
    \item \href{https://energy-web-foundation.gitbook.io/energy-web/technology/the-stack/trust-layer-energy-web-chain/system-architecture/system-contracts/block-reward-contract}{\textbf{Reward Contracts}}: gestisce la ricompensa dovuta al validatore
    \item \href{https://energy-web-foundation.gitbook.io/energy-web/technology/the-stack/trust-layer-energy-web-chain/system-architecture/system-contracts/holding-contract#overview}{\textbf{Holding Contracts}}: distribuisce i \gls{ewt} iniziali
\end{itemize}

\section{Costi di transazione}
Una transazione è una qualsiasi operazione che modifica lo stato della blockchain. Trasferire token fra account, creare un nuovo smart contract o modificarne lo stato sono esempi di transazioni. \\
Ogni transazione ha un costo che sarà pagato in \gls{ewt}.
Grazie alla specificità di utilizzo della \gls{ewc} e all'algoritmo di consenso, che insieme contribuiscono a ridurre la congestione della rete aumentandone il throughput,
le tariffe sono generalmente molto basse e vanno da $0.00001$ a $000000.1$ \gls{ewt} \cite{art:manage-costs}. \\
Il costo monetario da pagare per poter effettuare una transazione lo si può calcolare a partire dai seguenti fattori:

\begin{itemize}
    \item \textbf{Il costo in gas:} il gas rappresenta la complessità computazionale necessaria per risolvere la transazione. Più è complessa l'operazione più gas sarà richiesto
    \item \textbf{Prezzo del gas:} valore dell'unità di gas in EWT. Prima di effettuare una transazione, l'utente stabilisce quanto è disposto a pagare per unità di gas. Più è alta l'offerta, maggiore sarà la priorità della transazione
    \item \textbf{Valore di mercato del token:} se si vuole ottenere il costo della transazioni in moneta fiat, basta effettuare la conversione fra EWT spesi e il loro valore monetario
\end{itemize}

Riassumendo con una formula \cite{wiki:ew-transaction-cost}: \\
$ costo(\$) = costo\ gas(gas) * prezzo\ gas(token/gas) * valore\ token (\$/token) $.

\subsection{Precise Proofs}
Precise Proofs è uno schema basato sugli alberi di Merkle (cfr. \autoref{sec:merkle-tree}), in grado di verificare l'autenticità della struttura dati rivelando solo un subset di dati. \\
Si parte realizzando un albero di Merkle a partire dai dati interessati e si rende pubblica la radice. \\
Un terzo ente, interessato a verificare la validità di un subset di dati, riceverà solo i dati effettivamente richiesti e le hash dei dati non necessari.
In questo modo si hanno abbastanza informazioni da poter ricostruire la radice minimizzando i dati resi pubblici.

\subsection{Zero Knowledge Proof Protocols}
Gli Zero Knowledge Proof Protocols sono una categoria di protocolli con lo scopo di validare una computazione senza rivelarne alcun input (es. validare una transazione senza rivelare mittente, destinatario, movimento etc.).

\subsubsection{zkSNARKs}
zkSNARKs è probabilmente il protocollo più avanzato in questa categoria, sviluppato dal team Zcash \cite{wiki:zkSNARKs}.
Sebbene sia il più maturo, presenta un tallone d'Achille non indifferente. \\
Per poter utilizzare la Zero Knowledge Proof è necessario realizzare un circuito aritmetico che rappresenta la computazione da provare e generare da quello una coppia di chiavi prover/verifier.
La creazione di questo setup deve essere sicura e pattuita a priori, perché se dovesse essere compromesso un avversario potrebbe approfittarne per realizzare prove false.

\section{Storage}
Il fatto che tutti i dati siano immagazzinati sulla blockchain per essere accessibili si traduce in un incremento della memoria necessaria per salvare l'intera blockchain su un dispositivo fisico,
in una maggiore congestione della rete con conseguente calo di prestazione e in un aumento di costi possibilmente evitabili nelle esecuzione degli smart contract. \\
Una prima soluzione è di evitare di immettere dati sulla blockchain, e limitarsi ad utilizzare gli hash degli stessi, per poi limitarsi a verificarne la validità off-chain. \\
Nei casi in cui il salvataggio dei dati fosse necessario, si può provare ad optare per una delle tante soluzioni che offrono un servizio di storage distribuito. \\
Le principali soluzioni sono \gls{ipfs} e Storj.
Per una lista completa è possibile consultare la documentazione \cite{wiki:ew-storage}

\subsection{InterPlanetary File System (IPFS)}
\gls{ipfs} \cite{prod:ipfs} è un modello di filesystem distribuito.
Invece di indirizzare i file con la loro locazione, li si identifica con un hash ottenuto dal loro contenuto.
Se il file è troppo grande, lo si spezza in più parti con lo stesso risultato. \\
I nodi che fanno parte di questa rete si auto-iscrivono in delle tabelle di \gls{dht} che tengono traccia dei nodi che possiedono il file associato ad uno specifico hash. \\
Se si vuole contribuire alla rete, una volta scaricato il file lo si tiene in cache, fornendolo ad altri utenti che lo richiedano. \\
Il fatto che un file sia identificato dal suo hash garantisce inoltre l'integrità del dato e la sua immutabilità.
Per aggiornare un file, infatti, bisogna utilizzare il sistema di versioning previsto dal protocollo. \\ \\

\subsubsection{\gls{vc} su \gls{ipfs}}
Nell'ecosistema di \gls{ew} le \gls{vc} sono memorizzate su \gls{ipfs} come hash. 
Il riferimento viene mantenuto dal \gls{did} document corrispondente attraverso un service endpoint, previsto nello standard \cite{wiki:did}. 
In altre parole, i service endpoint sono ciò che collega il \gls{did} document alle proprie \gls{vc}. \\
Per poterle leggere, le \gls{vc} devono essere quindi ottenute da \gls{ipfs} e decodificate. \\
\gls{ew} fornisce una libreria javascript, \href{https://github.com/energywebfoundation/ew-did-registry/tree/development/packages/claims}{\textit{claims}}, in grado di assolvere proprio questa funzione.

\subsection{Storj}
Storj \cite{prod:storj} è molto simile ad IPFS, con l'incentivo che i nodi sono pagati per svolgere la loro funzione di content-storage.
Ovviamente il contratto prevede che l'host sia in grado fornire su richiesta in qualsiasi momento tutti i file che afferma di possedere.

\section{Governance}
La natura decentralizzata della blockchain rende un qualsiasi cambiamento che riguarda la sua infrastruttura un problema non banale. \\
Per gestire al meglio i possibili aggiornamenti che la \gls{ewc} potrebbe necessitare, si è deciso di affidarsi a chi la comprende bene, cié gli sviluppatori.
Il diritto di voto che determina quali proposte di modifica vengano accolte e quali respinte sarà determinato dalla quantità di gas che un singolo sviluppatore riesce a "generare" con i propri smart contracts.
L'idea è che il valore che lo sviluppatore fornisce all'intero sistema rappresenta il suo peso nella votazione \cite{wiki:ew-governance}. \\
Ciò non esclude che, in caso si renda necessario, alcune scelte troppo complesse possano essere prese da persone designate senza passare per il meccanismo di voto della blockchain.

\begin{figure}[h]
    \includegraphics[height=10cm,keepaspectratio]{ew-governance}
    \centering
    \caption{Passi per l'approvazione di una modifica all'infrastruttura di \gls{ewc} \cite{img:ew-governance}}
    \label{lab:ew-governance}
\end{figure}

\section{Staking}
\label{sec:staking}
Per fornire quello che nell'architettura di ES-DOS è chiamato "Utility layer", è necessario il contributo di operatori che mettano a disposizione le proprie risorse a chi ha intenzione di sviluppare un servizio.
Perché il tutto sia affidabile e adatto ad un ambiente di produzione, devono essere garantiti dei livelli di servizio, in maniera analoga ad un qualsiasi \gls{sla} proposto da un fornitore di SaaS. \\
La soluzione proposta non è dissimile a quella implementata in altre blockchain, e si rifà al modello \gls{pos}.
Gli attori in questo modello vengono suddivisi in due categorie:

\begin{itemize}
    \item \textbf{Service providers:} sono organizzazioni che mettono a disposizione i nodi di utility.
          Per essere approvato, un service provider deve poter dimostrare la propria identità e depositare in garanzia una quantità di EWT per un periodo multi-annale.
          Dopo essere stati approvati, i service providers possono aggiungere ulteriori nodi di utility incrementando proporzionalmente il deposito di EWT.
          Finché lo \gls{sla} continua ad essere rispettato, i service providers guadagneranno un interesse in base al loro deposito.
          In questo momento, il deposito minimo necessario per essere approvati come service provider si attesta fra i $10000$ e i $100000$ EWT, a cui si aggiungono fra i $1000$ e i $10000$ EWT per service node \cite{art:ew-staking}.
    \item \textbf{Patrons:} sono gli individui o le organizzazioni che finanziano un service provider depositando dei loro EWT per il service provider.
          Differentemente rispetto a ciò che accade per i service provider, non c'è un minimo alla somma depositata e questa può essere ritirata in qualsiasi momento.
          Questo porta i patrons a favorire service provider che rispettano gli \gls{sla} e che offrano il miglior modello di revenue-sharing.
\end{itemize}

\begin{figure}[h]
    \includegraphics[height=10cm,keepaspectratio]{ew-staking}
    \centering
    \caption{Come l'utility layer supportato dallo staking si integra in EW-DOS \cite{art:ew-staking}}
    \label{lab:ew-staking}
\end{figure}