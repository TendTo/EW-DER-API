\chapter*{Abstract}
\addcontentsline{toc}{chapter}{Abstract} % add the chapter to the index

%%%%%%%%%%%%
% Content
%%%%%%%%%%%%
Il mondo della produzione e distribuzione dell'energia elettrica sta andando incontro ad un processo di decentralizzazione sempre più rapido. \\
La spinta nasce dall'ingresso nel mercato di nuovi piccoli produttori di energia rinnovabile con capacità variabili e
da una domanda che necessita di sempre più flessibilità e con una rinnovata sensibilità ambientale. \\
Per poter supportare questo nuovo tipo di mercato, incentrato su un rapporto più diretto fra produttore e consumatore,
diversi progetti hanno preso in considerazione la tecnologia delle blockchain, sfruttandone i punti di forza e cercando di superarne le limitazioni. \\
L'utente ottiene anche un controllo maggiore ha sulle informazioni che fornisce, ottenendo inoltre le garanzie di integrità e non ripudio che le tecnologie
crittografiche combinate con la blockchain sono in grado di fornire. \\
In questo documento ci si concentrerà particolarmente sulle soluzioni in questo ambito proposte dal progetto Energy Web,
al fine di fornire un esempio concreto di una possibile implementazione. \\
I concetti chiave possono comunque essere applicati ad altri progetti della stessa natura.